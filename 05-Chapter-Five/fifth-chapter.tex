This thesis examines the political behavioural change of local legislators in their legislative voting and parliamentary questions asked to the executive government (The Executive Yuan) after an electoral reform. Specifically, the causal impact of the 2008 reform of the Taiwan electoral system is carefully examined, combined with various newly developed and innovative approaches. The findings of the thesis contribute to the large body of literature on the origins and consequences of electoral systems.

The most direct consequence of Taiwan's electoral reforms is that it exemplified the impact of electoral changes on party relations and individual legislative behaviours. To put it differently, SNTV made it easy for extremists to be elected without winning a large share of votes  \citep[e.g.,][]{Cox1987, Cox1990, Cox1996, Cox2008, Carey1995}. In particular, SNTV allowed the district magnitude to exceed one and led to a dispersed distribution of within-party ideology spectrum \citep[e.g.,][]{Stockton2010, Carey1995, Catalinac2019, Catalinac2017, Catalinac2016}, as well as an increased number of candidates who run on personal reputation and regional organisations. 

Aware of the side effects of SNTV, mainstream parties jointly initiated an election reform, aiming to reduce co-partisan conflicts and improve the political atmosphere in congress. The focus of this reform was expected to improve partisan conflicts within congress by contracting the number of legislators to be elected. In addition, the literature anticipates that under this system, electoral competition is winnowed down to two parties \citep[e.g.,][]{Catalinac2017, Downs1957, Duverger1954, Merrill2002, Magar1998, Reed2001}. Therefore, SMD was chosen as the desirable electoral system in lieu of the SNTV \citep[e.g.,][]{Yu2008, Liao2013, Jou2009}.

In general, this thesis investigates how legislators position themselves in response to an electoral system from the single non-transferable vote to single-member districts. Three main chapters that analyse historic legislative archives are developed extensively to demonstrate the impact of the reform on legislators' representation and preference, using several new and rapidly evolving estimation methods. To narrow down, the thesis focuses on the following three research subjects: Does electoral reform mitigate intraparty competition and political polarisation between parties? Does the reform decrease legislator's intention to particularistic policies? In what follows, I describe the structure and briefly summarise the findings and contributions of the thesis.

In the \autoref{chap:ch2} , I utilise the ideal point estimation to study the impact on legislators' preferences in relation to their party. The empirical evidence shows that intraparty fractionalisation increases while ideological differences between major parties are drastically polarised in the SMD. Controlling for yearly effects during the presence of the reform, however, I find that the impact of party division decreases as time goes by. In \autoref{chap:ch3}, I investigate how the reform changes legislator behaviour and tendencies toward constituencies. I investigate this topic using written parliamentary questions through an electoral reform from multi-member districts (MMD) to single-member districts (SMD). I find that legislators under SNTV are more likely to express political intention about pork barrel projects in written parliamentary questions. However, the institutional change subsequently demonstrates heterogeneous effects on mainstream parties and small parties, respectively.  

Last, I analyse intergovernmental transfers allocated to each municipality to explore how pork barrel spending is politically motivated by experienced mayors who used to serve as legislators in the Taiwanese Congress. In \autoref{chap:ch4} , I find that municipalities whose mayors had a longer career spent in the legislature are more likely to receive higher fiscal expenditures. The effect is even more substantial if the mayors have worked in legislative standing committees. On the contrary, mayors' prior political career in ministries does not significantly help their municipalities obtain the grant. The findings suggest that compared to experience as a central government official, their legislative career significantly impacts the distribution of the transfers to municipalities.

Overall, the reform may not immediately reduce intraparty conflicts but shortly exacerbate the inter-party relation when the new system is introduced. In addition, legislators' incentives for mainstream parties to run on personal votes significantly decreased, while the attention to general interest policies momentarily increased after the reform.

\section*{\centering Challenges and Limitations}

The thesis comes with several limitations. In \autoref{chap:ch2}, I demonstrate how the reform impacts legislators' ideological preferences and inter- and intra- party distance. However, since different topics were discussed in each session of roll call, the biggest challenge is to control for the session effect; i.e. there are heterogeneous effects across sessions that potentially affect the legislators' positioning. Specific topics related to political turmoils due to unforeseen incidents (e.g., Taiwan Strait Crisis in 1996, Million Voices against Corruption President Chen Must Go in 2006 and 2014 Sunflower Student Movement, 2019–20 Anti-Extradition Law Amendment Bill Movement in Hong Kong) and unexpected natural disasters (e.g., Typhoon Morakot, 1999 Jiji Earthquake, 2011 Fukushima Nuclear Disaster and Earthquake) are more likely to be discussed in current sessions. Therefore, these political incidents tend to polarise the main parties and disunite the distance between legislators and their parties. 

The best solution to tackle this issue would be to control for specific topics\footnote{i.e., the issues regarding the cross-strait and independence–unification.} in each session by manually including those incidents as much as possible as dummies in the econometric regressions. However, identifying and differentiating these contents one by one is exceptionally challenging. To this end, I address the problem concerning the heterogeneous sessional effects by including year dummies that can approximate the sessional effects. In \autoref{chap:ch3}, the effect of electoral reform is statistically significant in the specification of ideological dispersion between and between parties, suggesting that electoral reform increases the dispersion between parties between KMT and DPP, even after year dummies and demographic attributes are directly controlled in regression.

Second, the number of questions over years analysed in \autoref{chap:ch3} steadily decreases from 2003. This is due to the fact that the reduced amount of questions is potentially correlated with changes in media users and the occurrence of social media. In recent years, social media, such as Facebook and Twitter, has become a major platform for legislators to communicate with their voters. Therefore, legislators may utilise other means to view their points in relation to constituent concerns, such as Facebook.

Another limitation arises from the language transformation and its variation over time. As noted in \autoref{chap:ch3}, the training data of the pork barrel legislation were annotated by Dr Ching Jyuhn Luor \citep{Luor2008, Luor2009} between 2007 and 2009. That is to say, the deep learning architecture is only trained by learning to identify the pork barrelling features in the legislation in a given limited period, which potentially fails to comprehensively discover implicit notions invented after 2009. With BERT's self-attention mechanisms, the BERT-based framework may assist the machine classifier by simulating a similar skill set to the human brain that can identify more complex or unseen concepts derived from the labelled data to understand the underlying pork barrel features.

% With insightful literature  on the origins and consequences of electoral systems (e.g. Huang 2017; Göbel 2012; Carey and Shugart 1995; Hsu and Chen 2004; André, Depauw, and Shugart 2014; André, Freire, and Papp 2014), the thesis attempts to estimate the  changes of each legislator on their legislative voting and parliamentary questions asked to the executive government (The Executive Yuan). 
% The numerous studies have described that voters' ideological positioning along the left-right ideological spectrum in Taiwan is largely determined by the stances of each major parties (KMT and DPP) that takes on the issue of cross-strait relations \citep[e.g.,][]{Clark2012, Fell2004, Huang2017}. 
% Disrupted Careers leading to a subnational shift in particularism by ex Congressional mayors retaining connection to national politics


\section*{\centering Possibilities for Future Research}
First of all, understanding the positions of parties and legislators is fundamental to conceptualising party cohesion and their representation in most democracies. With parliamentary questions and topics classified by the official website,\footnote{The website of the Legislative Yuan} I can use the data set to estimate intraparty heterogeneity and variability in issue attention by looking at what and why legislators are more likely to oversee ministry officials on one specific topic than others. I can estimate the position of each legislator on the left-right dimensions using the keywords extracted from the questions that legislators request for information on policies and activities of ministerial officials. This is important to see how the electoral reform shapes legislators' issue attention and varies across time and the different electoral systems. The findings may shed new light on providing a different approach to measuring party cohesion and understanding changes in political behaviour in representation and political accountability through electoral reform.

Second, the work by \citet{Catalinac2016} has looked at the relationship between electoral reform and the behaviour of legislators covering pre- and post-reform periods by analysing the election manifesto using the Latent Dirichlet Allocation topic model. The paper successfully identified that the Liberal Democratic Party (LDP) candidates in the SMDs expressed more programmatic policies, such as national policies, and promised fewer pork barrel goods to the district. Likewise, I can apply a similar approach to clustering by discovering the variation of words frequently used when Taiwan's reform occurred. This method offers the advantage of combing through massive amounts of text in extensive document collections without going through the content in advance. 

However, topics extracted by the LDA method is generally not guaranteed to be well readable, especially in an unspaced language like Mandarin.\footnote{LDA substantially identifies each document as the mixture of topics by describing a distribution of words but assuming each word's equal importance. Mandarin is one of the unspaced languages, meaning it is written without spaces between characters and words.} Therefore, the performance of topic terms in each classification group may not provide explicit information if all part-of-speech (POS) tags are included. Particularly when handling the context of Taiwan politics, we need extra caution when text preprocessing, as there are some special terms, such as names of politicians and political institutions, as well as collocations.\footnote{In the thesis, I will train my language model (parser) based on an existing model to improve performance and classification prediction in Chinese.} To resolve this, I will manually augment the current pre-trained language model (UDPipe Traditional Chinese GSD 2.8 created by \cite{Straka2016, Straka2017}) for the POS task by enriching the training text data following the structure of the CONLL-U format. Once having these tags and linguistic features, I can extract a set of keywords and remove irrelevant the POS tagset such as auxiliary tokens, punctuation and irrelevant interjections and conjunctions.

In addition, previous works by \citet{Yu2013, Martin2015, Lau2014} suggest the coherence score (used to measure the quality of the topic model) can be improved by adjusting the elements of part of speech tags \textemdash limiting the corpus to nouns. Considering the unspaced feature in Mandarin, the future project will follow a similar approach to generate the LDA topic model by keeping specific POS tags in Chinese (such as nouns, verbs and adjectives), which is more efficient for summarising the multitude of topics and improving semantic coherence.\footnote{This approach can be utilised to extract more informative linguistic features and noun phrases in Chinese.}

In the third place, I anticipate combining estimates from legislative roll calls and parliamentary debates. Therefore, I can parallelly calculate each legislator' the difference and variance between political positions derived from roll calls and words expressed in the debates on salient topics such as same-sex marriage and cross-strait issues across the legislative sessions. This intrigues me if I can explain why and how legislators present their preferences consistently in both voting and speeches while some do not. I would like to see some legislators' voting preference correlates with their speech ideal points whereas some of them do not. 

Last but not least, this thesis contributes some applicable codes and packages for reproducibility to the open-source community and academia. For textual data, I have written a Python program (\href{https://davidycliao.github.io/legisCrawler}{davidycliao.github.io/legisCrawler}) for scrapping parliamentary documents on the official website of the Legislative Yuan (Taiwanese Congress) and attempted to implant an additional module for getting a corpus of parliamentary debates. For this purpose, I also have developed an R package (\textit{LegisTaiwan} \href{https://davidycliao.github.io/legisCrawler}{davidycliao.github.io/LegisTaiwan}) that requests voting records of legislation via the official API of the Taiwan Legislative Yuan Library. Further, \autoref{chap:ch3} looks at the proportion of parliamentary questions devoted to particularistic goods. To measure these quantities, I have trained a dynamic convolutional neural networks on top of the BERT model to identify pork-barrel features. The source code is written in Python using TensorFlow 2.4 and made available on the repository (\href{https://github.com/davidycliao/PorkCNN}{github.com/davidycliao/PorkCNN}). The application of this repository can be extended to other classification tasks (i.e., hate speech and extremism on the Internet). Also, it can be used for similar research (i.e. identifying the media post of the legislator on Facebook) that aims to quantify pork barrel characteristics in the context of Taiwan politics using Traditional Chinese. 

However, those codes and packages were created in 2019, 2020 and 2022, respectively. For codes written at different times, their functions and modules only target a specific type of data analysis to satisfy a certain purpose. Therefore, the original layout in the packages accumulates excessive expansion in each module, leading to issues of feature creep and making those programmes hard to maintain simultaneously. With recent advancements in the integration of Python and R, those packages are required to be rewritten and integrated into the same building block.
