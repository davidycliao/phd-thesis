
\section*{\centering The Origin of Taiwan's Electoral Reform}
% Members more dependent on party performances -> Becomes more polarized
% Less pork focused for the large parties (however, small parties stay pork focused)
% Compare to Japan case
% Disrupted Careers leading to a subnational shift in particularism by ex Congressional mayors retaining connection to national politics
% See latin american cases with subnational careers post legislative, see other municipal pork connection literature (probably Brazil relevant in both aspects)
% The intensity of (some of?) these effects diminish over time because …. (also deal with this in intro/conclusion, as well as interpreting each chapter)

During the past two decades, numerous Asian democracies, such as Japan and Southern Korea, have reformed their electoral systems. These reforms were enacted in pursuit of a common goal: improving the quality of political systems and representation. Taiwan is not an exception. The Single Nontransferable Vote (SNTV) has been used for every election, including the local communities, the provincial councillors, and the National Assembly Election. In SNTV, each voter from the district only casts one vote for a single candidate. More importantly, the votes received by candidates cannot be transferred within co-partisans. This nontransferable feature of votes increases candidates' incentive to initiate regional groups (such as the local irrigation association 水利會 \textit{shuilihui} and a vote-broker 樁脚 \textit{zhuangjiao}) to reinforce electoral supports locally. To pass the minimum threshold of being elected, candidates have strong incentives to run on the personal vote and maximise their own votes by behaving rebelliously against their party to please parochial supporters. 

Numerous reasons aligned the KMT (Kuomintang, the Chinese Nationalist Party) politically with the DPP (Democratic Progressive Party) with regard to the agenda of promoting electoral reform. During the transition from an authoritarian state to democracy since the 1990s, co-partisan competitions made it extremely difficult for the KMT to maintain incumbent advantages in municipal and assembly elections.\footnote{From 1949 to 1987, the KMT Kuomintang was the only major ruling party, while Taiwan was an authoritarian one-party state during the period of martial law.} After the 1994 election fiasco, the Chief of the Executive Yuan (led by the KMT administration), Lian Zhan (連戰), publicly advocated a two-vote system for a single constituency but was opposed by most rival parties. In June 1996, the Central Election Commission proposed a compromised plan that reduced the number of seats and introduced a two-vote system. On August 23rd 2004, supported by the opposition party DPP, the Fifth Legislative Yuan held an extraordinary session and passed the Seventh Amendment to the Republic of China Constitution.\footnote{On March 13 2008, the People's Anger Action Alliance (人民火大行動聯盟) initiated the Campaign ``Do Not Vote for Ma and Xie''. They raised concerns against the majority parties, the DPP and the KMT, which have jointly monopolised the legislative election and annually received over NTD 1 billion for the election campaign subsidies from the national treasury. They accused the majority parties of passing particularistic legislation together, citing the reform yielding the effect of reduced legislative seats as an example.} This Amendment for National Assembly Election not only reformed the district electoral system by introducing a single-member district but also reduced original seats from 225 to 113. The KMT claimed that the electoral reform away from SNTV was in line with the world trend.

SNTV was adopted in Japan and South Korea. Yet, this electoral system has still been used for national legislative elections in many countries (e.g., Puerto Rico, Indonesia, Chile, Jordan, Libya). More recently, the Legislative Council of Hong Kong reformed the original electoral system, simple plurality, to SNTV in 2021. Empirical examinations of the drawback concerning SNTV have asserted for a few decades that the system not only increases factional politics and clientelism \citep[e.g.][]{Chang2007, Wu2003} but decreases candidates' loyalty to their party as well \citep[][]{Reed2003, Herron2018, Hsu2004, Nathan1993}. Some studies have found that legislators elected in SNTV are prone to engage less in particularistic legislation \citep[e.g.,][]{Lancaster1986, Crisp2004b, Kerevel2015, Rogowski2017} and constituent service \citep[e.g.,][]{Heitshusen2005}, whereas other literature focusing on Japan's electoral reform finds that legislators elected by SMD report declining incentives to bring home the bacon. For example, \citet{Hirano2006} has discovered that Japan's pork barrel spending, such as intergovernmental transfers, appears to be more concentrated in the LDP candidate's district under SNTV than SMD.\footnote{Moreover, \citet[][]{Sheng2014} finds that the proportion of the legislation delivering general benefits decreases in SMD while a higher proportion of legislation is implanted with pork barrel projects under SNTV. This finding is consistent with \citet{Catalinac2016}, which analyses Japan's election manifestos, that the electoral reform in Japan was associated with a decline in promised provisions of particularistic goods and an increase in programmatic policies.} Nevertheless, I need more granular observational data recording daily legislators' motions, such as parliamentary questions and legislative votes, to evaluate how the reform impacts the development of political parties and governance in the legislature. 

This thesis aims to evaluate the concurrent effects of the reform and illustrate how institutional change could unexpectedly create disruptive political sentiments by answering the following research questions: Does the 2008 reform from SNTV to SMD serve the goal of mitigating political chaos within the congress? Does the reform reduce legislators' incentives to run on the personal vote and pork barrel spending of sorts, thus, increasing their effective performance to provide public goods? If so, why are some municipalities still disproportionately rewarded with more distributive spending? In this thesis, I concretely investigate how legislators changed their political behaviours in response to the 2008 electoral system. These three research topics can make use of historic legislative data to measure and analyse the impact on legislators' representation and positioning in Taiwan. The archives analysed in the thesis include the data set for legislative roll call used to measure intraparty and interparty competition from 1993 to 2016, parliamentary questions asked by legislators from 1993 to 2020, and intergovernmental transfers allocated across municipalities (sub-national units) to analyse distributive spending covering a period before and after the electoral system.

With a more granular and novel data set that records legislators' behaviours, the adoption of the new electoral system in Taiwan provides a natural quasi-experiment to evaluate how the reform shapes legislators' electoral strategies and ideological preferences. To my knowledge, the literature analysing the impact of the electoral systems on political behaviour could benefit from quantitative research that investigates how a reform from SNTV to SMD impacts the legislator's representation by analysing legislative roll calls and parliamentary questions covering pre- and post-reform periods. This is largely due to data availability and the under-development of appropriate analytic approaches serving political science at the time. In recent years, partisan conflicts between major parties have intensified, and the intraparty relationship has become increasingly tense after reform. In studying the relationship between the reform's consequences and partisan competitions, the thesis applies ideal point estimation and natural language processing techniques that are currently deployed to study legislatures and computational social science. Therefore, it strengthens the understanding of Taiwan party politics and party competition inside Taiwanese congress (the Legislative Yuan). 

\section*{\centering The Implication}
This thesis documents several key contributions made to the fields of Taiwan's legislative politics and electoral system. Estimating the policy positions of parties and legislators is a foremost step to understand electoral competition and party formation. The ideal point estimation of binary choice has revealed numerous insights into the structure and dynamics of legislative voting \citep[e.g.,][]{Carroll2013, Gray2019} and judicial preferences \citep[][]{Martin2007, Epstein2007}. Few studies have examined legislators' voting behaviour that includes a more extended period covering an electoral reform through SNTV to SMD, and looked at how reform influences the way legislators ask questions. The 2008 reform in Taiwan has allowed us to make inferences based on the actual impact of electoral changes on party relations and individual legislators' behaviours. In \autoref{chap:ch2} and \autoref{chap:ch3}, I utilise legislative voting records and parliamentary questions to study how legislators changed their focus and ideological positions after the electoral reform.

With increasingly available political data, the application of the classification task has received great attention in political science \citep[such as][]{Chatsiou2020, Engel2017}. In \autoref{chap:ch3}, I contribute to the literature on electoral systems and political representation by demonstrating how institutional change decreases legislators' incentives to capitalise on their personal reputation by adopting electoral strategies that target the median voter. Combining a dedicated deep learning algorithm with robust regression analysis, I estimate the impacts of electoral reform from SNTV-MMD to SMD on legislative behaviour. This chapter uses the Pork-barrel Legislation Dataset to train deep learning models to detect pork-barrel features in parliamentary questions over time. I find that SMD motivated small-party legislators to make more effects by targeting parochial groups of voters through asking more pork-barrel questions, while majority parties pay more attention to general-interest questions to attract more median voters. 

The third advancement is studying the indirect effect of electoral reform on pork barrel spending allocated to sub-national entities, i.e. municipalities. Numerous studies on distributive politics in Taiwan focus on explaining the effect of legislators' internal privileges, partisan characteristics, and electoral vulnerability on the distribution of particularistic spending across districts \citep[e.g.,][]{Luor2000, Luor2004, Luor2008, Lai2013} but few examine the effect exerted by municipal officials. Due to the reduced number of seats after the reform, many legislators changed careers to become municipal mayors. In \autoref{chap:ch4}, I explain how the reform's effects of reduced seats unexpectedly exacerbates the disproportionate allocation of intergovernmental transfers across municipalities. I also find that mayors with more experience in congress and standing committees will receive more intergovernmental transfers than others.

In sum, the thesis makes three distinct contributions to the literature on Taiwan legislative politics and beyond: 1) it provides new analytic techniques developed by computational social science to comprehend party competition and legislative politics in Taiwan, 2) it applies natural language processing and text classification task in political science to re-evaluate pork barrel programs in the context of Taiwan politics, and 3) it analyses how the electoral reform affects legislators' behaviour and communication strategies via asking parliamentary questions.

\section*{\centering Plan for the Thesis}
The thesis begins with the premise that Taiwan electoral reform is used to improve intraparty conflict, interparty competitions and distributive politics. The overall explanation is that the reform conditionally alleviates intraparty competition. Subsequently, legislators' incentive to run on personal votes decreases while attention to general policies increases after the reform. The following three chapters present theoretical arguments and empirical evidence to support my claim. In recent years, the item response theory-based scaling measure of legislative voting developed by \citet{Clinton2004} has been successful in its application to understanding legislative voting behaviour \citep[e.g.][]{Zucco2011, Tsai2020, Gray2019}. In \autoref{chap:ch2}, I 
investigate the strategic (inter- and intra-) party positioning in response to an electoral system transition from SNTV to SMD by estimating individual legislators' ideological positions using sessional roll call votes covering pre- and post-reform periods. Legislators in this chapter are subjects of interest with a latent level of ability measured by varieties of roll calls  (test items) and political ideology. The IRT measurement can be utilised to set up item difficulty parameters to discover estimates of the voting preferences of legislators and expressed political views in parliamentary speeches. The finding suggests that during the electoral transition from SNTV to SMDs, ideological positions between two major parties are drastically polarised. Meanwhile, this electoral transition significantly disunifies co-partisan legislators on the ideological spectrum. 

In \autoref{chap:ch3}, I offer empirical evidence showing that Taiwan's electoral reform reduces legislators' attention to the pork barrel projects. In this chapter, I train deep learning models on multi-convolutional neural networks with an embedding layer extracted from one of the most powerful Transformer architecture \textit{BERT} to quantify pork-barrel features of parliamentary questions across time. With Transformers' self-attention mechanisms, this combination approach enables my pork-barrel algorithm to learn more condensed features of embedding representation and better handle polysemous words commonly seen in Mandarin than traditional embedding approaches like Word2Vec and Glove. In addition, the technique utilised in chapter \autoref{chap:ch3} not only takes less time to train the neural network model but generates similar prediction performance compared with ordinary convolutional neural networks and the BERT model, respectively. Evidence shows that the transition of electoral reform incurs essential changes in legislators' behaviour. Legislators under multi-member districts are more likely to express political interests regarding pork-barrel projects in the written parliamentary questions. 

In \autoref{chap:ch4}, I examine the link between pork barrel spending and the mayor's prior career in the context of Taiwan's electoral reform. After the electoral reform from the SNTV-MMD to the SMD, the intergovernmental transfer as a pork barrel with signalling purposes is no longer necessary for elected legislators to achieve electoral success. However, in reality, most Taiwan intergovernmental transfers are politically motivated and disproportionately distributed across municipalities. In chapter \ref{chap:ch4}, I find that municipalities whose mayors with more years of experience in the legislature are more likely to be allocated more transfers. The effect is even more substantial if those legislators were previously connected to the standing committees. 

Finally, as a first step to analysing recent drawbacks of the reform and its performance, this thesis aims to shed light on how the political agenda determined by the mainstream parties unexpectedly made Taiwan's political environment unstable during and after the transition. Taiwan's electoral reform aggravated intraparty competition during the transition, leaving the legislature with extreme polarisation at times. However, mainstream party legislators' incentive to run on personal votes significantly decreased while the attention to general interest policies increased after the reform. 


% Finally, this is not a thesis about criticizing Taiwan's electoral reform but sheds light on how the political agenda recklessly determined by the majority parties unexpectedly made Taiwan's political environment unstable during and after the transition. Taiwan's electoral reform aggravated intraparty competition during the transition, leaving the legislature in chaos at the time. However, legislators' incentive for mainstream parties to run on personal votes significantly decreased while the attention to general interest policies momentarily increased after the reform. 
